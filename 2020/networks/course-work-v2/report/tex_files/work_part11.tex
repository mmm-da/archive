Эта глава содержит 4 статьи:

\begin{itemize}
\tightlist
\item
  \textbf{Статья \(272\). Неправомерный доступ к компьютерной
  информации}
\item
  \textbf{Статья \(273\). Создание, использование и распространение
  вредоносных компьютерных программ}
\item
  \textbf{Статья \(274\). Нарушение правил эксплуатации средств
  хранения, обработки или передачи компьютерной информации и
  информационно-телекоммуникационных сетей}
\item
  \textbf{Статья \(274^1\). Неправомерное воздействие на критическую
  информационную инфраструктуру Российской Федерации}
\end{itemize}

Ни в одной из вышеперечисленных статей не описано наказание за
сканирование сегментов сети, но информацию полученную при сканировании
сети такую как: ОС и ее версию, открытые порты для приложений
прикладного уровня и их версии, можно использовать для совершения
преступлений в информационной сфере.

Так же провайдеры интернет услуг зачастую запрещают использовать сканеры
для сканирования портов в сети Интернет.

\hypertarget{ux440ux430ux437ux440ux430ux431ux43eux442ux43aux430-ux43fux440ux43eux433ux440ux430ux43cux43cux44b}{%
\section{Разработка
программы}\label{ux440ux430ux437ux440ux430ux431ux43eux442ux43aux430-ux43fux440ux43eux433ux440ux430ux43cux43cux44b}}

\hypertarget{ux432ux44bux431ux43eux440-ux438ux43dux441ux442ux440ux443ux43cux435ux43dux442ux43eux432-ux440ux430ux437ux440ux430ux431ux43eux442ux43aux438}{%
\subsection{Выбор инструментов
разработки}\label{ux432ux44bux431ux43eux440-ux438ux43dux441ux442ux440ux443ux43cux435ux43dux442ux43eux432-ux440ux430ux437ux440ux430ux431ux43eux442ux43aux438}}

В рамках данной курсовой работы был реализован TCP сканер портов, SYN
сканирование не было реализовано так как ОС Windows не разрешает
отправлять пакеты используя ``сырые'' сокеты (``сырые'' сокеты позволяют
напрямую изменять и просматривать содержимое IP пакета). Для реализации
SYN сканирования в ОС Windows потребовались бы привилегии администратора
в системе, а так же специальный сетевой драйвер позволяющий изменять IP
пакеты напрямую.

Сканер был реализован на языке С++ с вставками на языке C, а так же
использовании кроссплатформенного фреймворка Qt для создания
графического интерфейса и многопоточности.

\hypertarget{ux43fux440ux43eux435ux43aux442ux438ux440ux43eux432ux430ux43dux438ux435-ux433ux440ux430ux444ux438ux447ux435ux441ux43aux43eux433ux43e-ux438ux43dux442ux435ux440ux444ux435ux439ux441ux430}{%
\subsection{Проектирование графического
интерфейса}\label{ux43fux440ux43eux435ux43aux442ux438ux440ux43eux432ux430ux43dux438ux435-ux433ux440ux430ux444ux438ux447ux435ux441ux43aux43eux433ux43e-ux438ux43dux442ux435ux440ux444ux435ux439ux441ux430}}

Графический интерфейс, был спроектирован используя возможности
библиотеки элементов \textbf{Qt}. Эта библиотека позволяет создавать
программы с графическим интерфейсом, которые имеют одинаковое поведение
в разных ОС. А механизм слотов и сигналов позволяет быстро и эффективно
писать приложения с графическим интерфейсом.
