Результаты сканирования сети классифицируются следующим образом:

\begin{itemize}
\item
  открытый порт - сканером получен ответ, хост принимает соединения на
  данный порт.
\item
  закрытый порт - сканером получен ответ, хост не принимает соединения
  на данный порт.
\item
  заблокированный порт - сканер не получил ответ.
\end{itemize}

\hypertarget{ux442ux438ux43fux44b-ux441ux43aux430ux43dux438ux440ux43eux432ux430ux43dux438ux44f}{%
\subsection{Типы
сканирования}\label{ux442ux438ux43fux44b-ux441ux43aux430ux43dux438ux440ux43eux432ux430ux43dux438ux44f}}

Перед сканированием любого типа обычно проводится проверка на наличие
указанного хоста в сети. При помощи протокола ICMP отправляются echo
сообщения на все сканируемые адреса, но отсутствие ответа на echo запрос
не всегда означает отсутствие хоста в сети, так как системные
администраторы зачастую запрещают работу ICMP в целях безопасности.

Существуют следующие алгоритмы сканирования портов:

\hypertarget{ux441ux43aux430ux43dux438ux440ux43eux432ux430ux43dux438ux435-tcp-ux43fux43eux440ux442ux43eux432}{%
\subsubsection{Сканирование TCP
портов}\label{ux441ux43aux430ux43dux438ux440ux43eux432ux430ux43dux438ux435-tcp-ux43fux43eux440ux442ux43eux432}}

\begin{itemize}
\item
  \textbf{SYN сканирование} - самый распространенный тип сканирования.
  Сканер портов генерирует IP пакеты напрямую без использования сетевых
  функций ОС предназначенных для установки TCP соединения. Это все
  необходимо для того чтобы напрямую управлять содержимым заголовка TCP
  и позволяет не создавать полностью открытое соединение. Принцип работы
  SYN сканирования таков.

  \begin{enumerate}
  \def\labelenumi{\arabic{enumi}.}
  \item
    Сканер создает пакет с установленным флагом SYN и отправляет его на
    указанный адрес (диапазон адресов).
  \item
    Если порт на целевом хосте открыт то в ответ сканеру придет пакет
    SYN-ACK и это означает что порт открыт. Если хост не отвечает значит
    SYN сканирование скорее всего заблокировано на уровне правил
    межсетевого экрана.
  \item
    Сканер отвечает пакетом RST и закрывает соединение до завершение его
    установки.
  \end{enumerate}

  Этот способ сканирования позволяет одновременно сканировать большое
  количество адресов и портов не создавая большой нагрузки на хост и
  сеть созданием и закрытием множества TCP соединений. Но для работы
  такого сканера потребуются повышенные привилегии и стороннее ПО для
  генерации IP пакетов в обход TCP стека операционной системы.
\item
  \textbf{TCP сканирование} - самый простой способ сканирования портов.
  Используя сетевые функции операционной системы осуществляется попытка
  создать TCP соединение с хостом. При условии что порт открыт
  соединение будет установлено, иначе порт закрыт. Такой тип
  сканирования не требует специальный драйверов сетевых устройств и
  повышенных привилегий для сканирования сети, но сильно нагружает
  сканируемый хост.
\item
  \textbf{ACK сканирование} - этот тип сканирования используется для
  проверки наличия межсетевого экрана и определения сложности его правил
  фильтрации. На хост отправляются пакеты с установленным флагом ACK,
  который устанавливается только при установленном соединении. Если в
  межсетевом экране используются простые правила фильтрации то экран
  пропустит этот пакет, более сложные правила учитывают и блокируют
  такой тип сканирования.
\item
  \textbf{FIN сканирование} - данный тип сканирования использует
  особенность спецификации TCP (RFC 793), в которой описано что на пакет
  FIN отправленный на закрытый порт, сервер должен ответить пакетом с
  флагом RST, если порт открыт то сервер игнорирует такой пакет. Такое
  сканирование применяется если сервер умеет распознавать другие типы
  сканирования, но не все разработчики ПО придерживаются спецификаций
  RFC, поэтому FIN сканирование может не дать результатов.
\end{itemize}

\hypertarget{ux441ux43aux430ux43dux438ux440ux43eux432ux430ux43dux438ux435-udp-ux43fux43eux440ux442ux43eux432}{%
\subsubsection{Сканирование UDP
портов}\label{ux441ux43aux430ux43dux438ux440ux43eux432ux430ux43dux438ux435-udp-ux43fux43eux440ux442ux43eux432}}

В протоколе UDP отсутствует понятие соединение, поэтому мы не можем
определить силами протокола был получен отправленный пакет или нет.
Сканирование UDP порта имеет ряд особенностей. При отправке UDP пакета
на закрытый порт при включенном протоколе ICMP, сканер получит ответ
``порт закрыт'', если же системный администратор отключил ICMP то
компьютерам извне будет казаться что все порты открыты.

Для того чтобы обойти отключение ICMP или межсетевой экран можно
формировать UDP пакет специфичный для ПО работающего с данным портом.
Таким образом при открытом порте мы получим ответ уровня приложения. Но
подготавливать тестовые пакеты для всего сетевого ПО, работающего по UDP
- невозможная задача. Поэтому может использоваться комбинированное
сканирование - сначала сканируются все порты UDP отправкой пустого
пакета, а далее используются специализированные пакеты для выбранных
портов, если они поддерживаются ПО для сканирования.

\hypertarget{ux441ux43aux430ux43dux438ux440ux43eux432ux430ux43dux438ux44f-ux43fux43eux440ux442ux43eux432-ux432-ux437ux430ux43aux43eux43dux43eux434ux430ux442ux435ux43bux44cux441ux442ux432ux435-ux440ux444-ux438-ux43fux440ux430ux432ux438ux43bux430ux445-ux43eux431ux441ux43bux443ux436ux438ux432ux430ux43dux438ux44f-ux438ux43dux442ux435ux440ux43dux435ux442-ux43fux440ux43eux432ux430ux439ux434ux435ux440ux43eux432}{%
\subsection{Сканирования портов в законодательстве РФ и правилах
обслуживания
интернет-провайдеров}\label{ux441ux43aux430ux43dux438ux440ux43eux432ux430ux43dux438ux44f-ux43fux43eux440ux442ux43eux432-ux432-ux437ux430ux43aux43eux43dux43eux434ux430ux442ux435ux43bux44cux441ux442ux432ux435-ux440ux444-ux438-ux43fux440ux430ux432ux438ux43bux430ux445-ux43eux431ux441ux43bux443ux436ux438ux432ux430ux43dux438ux44f-ux438ux43dux442ux435ux440ux43dux435ux442-ux43fux440ux43eux432ux430ux439ux434ux435ux440ux43eux432}}

Уголовный правонарушения связанные с ЭВМ и сетями ЭВМ описаны в 28 главе
Уголовного кодекса РФ под названием ``\emph{Преступления в сфере
компьютерной информации}''.
