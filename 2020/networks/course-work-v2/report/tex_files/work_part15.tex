\begin{itemize}
\tightlist
\item
  реализация пула потоков в фреймворке Qt, при инициализации пула
  выделяется количество потоков равное количеству логических
  процессоров. При старте сканирования каждый поток сканирует 1 порт из
  диапазона общеизвестных портов, при завершении работы поток не
  закрывается, а используется для сканирование следующего необходимого
  порта.
\end{itemize}

\hypertarget{ux43eux43fux438ux441ux430ux43dux438ux435-ux43aux43bux430ux441ux441ux43eux432-ux438ux441ux43fux43eux43bux44cux437ux443ux435ux43cux44bux445-ux432-ux43fux440ux438ux43bux43eux436ux435ux43dux438ux438}{%
\subsection{Описание классов используемых в
приложении}\label{ux43eux43fux438ux441ux430ux43dux438ux435-ux43aux43bux430ux441ux441ux43eux432-ux438ux441ux43fux43eux43bux44cux437ux443ux435ux43cux44bux445-ux432-ux43fux440ux438ux43bux43eux436ux435ux43dux438ux438}}

Для отображения графического интерфейса, и обработки взаимодействий с
пользованием используется класс \textbf{SimplePortScan}.

В конструкторе класса создаются и настраиваются все элементы интерфейса
и их валидация.

\begin{Shaded}
\begin{Highlighting}[]
    \ExtensionTok{QRegExp}\NormalTok{ *re = }\KeywordTok{new} \ExtensionTok{QRegExp}\NormalTok{(}\StringTok{"\^{}(?:(?:25[0{-}5]|2[0{-}4][0{-}9]|[01]?[0{-}9][0{-}9]?)}\SpecialCharTok{\textbackslash{}\textbackslash{}}\StringTok{.)\{3\}(?:25[0{-}5]|2[0{-}4][0{-}9]|[01]?[0{-}9][0{-}9]?)$"}\NormalTok{);}
    \ExtensionTok{QRegExpValidator}\NormalTok{ *validator = }\KeywordTok{new} \ExtensionTok{QRegExpValidator}\NormalTok{(*re);}
\NormalTok{    ui{-}>setupUi(}\KeywordTok{this}\NormalTok{);}
\NormalTok{    ui{-}>progressBar{-}>hide();}
\NormalTok{    ui{-}>progressBar{-}>setMinimum(}\DecValTok{0}\NormalTok{);}
\NormalTok{    ui{-}>progressBar{-}>setMaximum(PORT\_EDGE);}
\NormalTok{    ui{-}>modeButton1{-}>hide();}
\NormalTok{    ui{-}>modeButton2{-}>hide();}
\NormalTok{    ui{-}>addressLine{-}>setValidator(validator);}
\end{Highlighting}
\end{Shaded}

и выбирается кол-во потоков в пуле.

\begin{Shaded}
\begin{Highlighting}[]
    \KeywordTok{this}\NormalTok{{-}>threads.setMaxThreadCount(}\DecValTok{400}\NormalTok{);}
\end{Highlighting}
\end{Shaded}

В слоте кнопки старта сканирования \texttt{on\_pushButton\_clicked()}
запускается пул потоков и в цикле добавляются задачи
\textbf{SimplePortScan} на сканирования конкретного порта (от 0 до
PORT\_EDGE). Так же запускается шкала процесса сканирования и она
отображается на экран.

Каждая задача \textbf{SimplePortScan} вызывает сигнал
\texttt{exit\_code(int,int)} указывающий на завершение сканирование
порта. он содержит номер сканируемого порта, и его состояние.

Обработчик этих сигналов \texttt{on\_result\_thread(int,int)} записывает
результаты в \textbf{QMap}
