\textbf{Задание:}

\begin{enumerate}
\def\labelenumi{\arabic{enumi}.}
\tightlist
\item
  Составить таблицу истинности системы булевых функций, которая состоит
  из трех функций \(f_1(X), f_2(X)\) и \(f_3(X)\), где
  \(X = {x_1, x_2, x_3, x_4, x_5}\). Булева функция f\_i(X) для k-го
  варианта определяется как \(f_i(X) = g_{k+i–1}(X) \land g_{k+3}(X)\),
  где \(g_j(X)\) --- булева функция, представленная в таблице 1
  (\emph{см. лабораторную работу № 1}) в строке j. Для составления
  таблицы истинности рекомендуется написать программу.
\item
  Получить систему минимальных дизъюнктивных нормальных форм булевых
  функций f1(X), f2(X) и f3(X).
\item
  Применить факторизационный метод синтеза многоярусной комбинационной
  схемы в базисе И-ИЛИ-НЕ с двухвходовыми элементами И и ИЛИ по системе
  минимальных дизъюнктивных нормальных форм булевых функций f1(X), f2(X)
  и f3(X).
\item
  Получить минимальную дизъюнктивную нормальную форму системы булевых
  функций f1(X), f2(X) и f3(X).
\item
  Применить факторизационный метод синтеза многоярусной комбинационной
  схемы в базисе И-ИЛИ-НЕ с двухвходовыми элементами И и ИЛИ по
  минимальной дизъюнктивной нормальной форме системы булевых функций
  f1(X), f2(X) и f3(X).
\item
  Написать программы, моделирующие работу схем, полученных в пунктах 3 и
  5, на всех входных наборах и строящие таблицу истинности каждой схемы.
  Сравнить полученные таблицы истинности с таблицей истинности исходной
  системы булевых функций.
\item
  Сравнить полученные в пунктах 3 и 5 схемы по Квайну и по
  быстродействию.
\end{enumerate}
